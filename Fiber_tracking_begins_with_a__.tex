Fiber tracking begins with a seed which serves as the starting point for the tracking procedure. Seeds can be placed anywhere in the brain, but are often placed near the boundary between the gray and white matter tissues. A seed can be tracked in either one or two directions. Tracking a seed in two, opposing directions allows us to identify the likely endpoints of a white matter track that passes along the seed point. The sequence of line segments that results from tracking a seed is usually called a streamlines. A collection of streamlines that follow similar paths is sometimes called a track. These track can represent white matter tracts of the brain, but it's useful to differentiate between the two because one is a biological structure while the other is a computational estimate of biology. The set of all streamlines produced by the process of seeding and tracking the whole brain is sometimes referred to as a tractogram.

Current tracking methods fall into two broad categories, deterministic fiber tracking and probabilistic fiber tracking. The main difference between these two broad categories is how the determine the tracking direction. Deterministic methods attempt to follow the optimal direction at each step, in this way the produce a "maximum likelihood" estimate of the underlying white matter structure. However, because small mistakes in tracking direction can aggregate along the length of a streamline, this methodology can be prone to missing pathways or truncating them early. Probabilistic tracking samples from a set of probable directions at each step. As a result each streamline represents a possible path through the white matter, but not necessarily the most likely path given the observed data.

Probabilistic tracking methods tend to produce a super set of streamlines that contain much of the likely white matter connectivity of the subject, but also many streamlines that represent true positives, meaning pathways reported as possible by the tracking algorithm which are not present the subject. Recent work in this field has proposed methods for identifying these false positive streamlines and eliminating them from further analysis \cite{Smith_2013}.

Fiber tracking can be used to produce a tractogram for the whole brain or a single track based on the choice of seeding. This track or tractogram can be used to identify either potential connections between regions of interest in the subject or the pathway that specific connections take. As diffusion MRI data sets become better, data sets with more multiple q-shells and more gradients become available and our ability to model this data improves, fiber tracking will allow us build more complex representations of the white matter connectivity in individual subjects.
