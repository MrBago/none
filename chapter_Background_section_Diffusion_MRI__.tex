\chapter{Background}
\section{Diffusion MRI Acquisition}
	Diffusion weighting in modern neuroimaging applications is most often done using the echo planer imaging (EPI) technique\cite{Poustchi_Amin_2001,Ordidge_1984}. Because diffusion MRI is attempting to measure the tiny, random motion of water molecules, it is highly sensitive to gross motion of the subject. EPI helps mitigate issues related to subject motion by reducing the time required to acquire each image to less than 100ms. This short acquisition time is also important in diffusion MRI because a complete diffusion MRI data set often includes dozens, or even hundreds, of brain volumes. The time efficiency of the EPI sequence makes it practical to collect complete diffusion MRI data sets in MRI scan sessions that are acceptable for research subjects and patients.
	
    The EPI sequence uses a series of gradient echos, called an echo train, to acquire a large portion of each image slice in a single radio frequency (RF) excitation. Multi-shot EPI sequences have been proposed in order to improve SNR and reduce distortions in diffusion MRI. However, these sequences require longer acquisition times and can produce phase errors in the the data\cite{Feinberg_1994}. The result of these limitations has been that single-shot EPI, where the entire image slice is acquired in a single excitation, has become dominant in diffusion MRI acquisitions. The diffusion weighting in a diffusion MRI sequence is introduced by using a pair of diffusion gradients on either side of a 180 degree RF pulse. The first of these diffusion gradients serves to dephase the MRI signal. The following 180 degree pulse inverts the signal so that the second diffusion gradient can now rephase the signal. In the absence of motion, these two gradients have no net effect on the diffusion signal. However, when magnetic particles move during this diffusion weighting procedure, their phases can only partially align, resulting in a single fallout. Equation \ref{eqn:diffSignal} gives the equation for the diffusion signal, $S$, defined in terms of $S_0$, the expected signal when no diffusion signal is applied \cite{2014}. In this equation, $\gamma$ is a constant, called the gyromagnetic ratio, $G$ is the gradient strength defined in units of Tesla per meter, $\delta$ is the length, in seconds, of each of two gradient pulses applied, and $\Delta$ is the time, in seconds, between the first and second diffusion gradient. Finally, $D(\vec{u})$ is the net diffusivity along the direction of the diffusion gradient, $\vec{u}$.

\begin{equation}
\label{eqn:diffSignal}
S(\vec{u}) = S_0e^{-\gamma^2 G^2 \delta^2(\Delta - \frac{1}{3}\delta) D(\vec{u})}
\end{equation}

In practice, all the terms that contribute to the diffusion weighting are grouped into one factor called the b value, where $b = \gamma^2 G^2 \delta^2(\Delta - \frac{1}{3}\delta)$. This b value has the inverse units of diffusivity, $\frac{s}{m^2}$, and represents the net diffusion weighting of the sequence. In some applications, like stroke imaging, diffusivity is so altered in the brain tissues that imaging with a few gradient directions is enough to observe and measure the effect \cite{Mukherjee_2000}. In stroke imaging a metric called apparent diffusion coefficient (ADC) is used. The ADC is the average diffusivity measured along three orthogonal directions, usually the x, y, and z gradient directions. The ADC can be estimated by acquiring three diffusion weighted scans and one scan with no diffusion weighting. However, to do any more complex modeling of the diffusion signal requires acquiring diffusion weighted volumes. The next section discusses different diffusion models, and the types of data need to support the modeling.
