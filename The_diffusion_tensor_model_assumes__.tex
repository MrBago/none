    The diffusion tensor model assumes that the diffusion of water molecules, in the tissues being imaged, can be described by a three dimensional Gausian distribution. This Gausian diffusion profile can be described by a 3x3 tensor, $\bold{D}$. Using this tensor, the diffusivity $D(\vec{u})$ can be written as $D(\vec{u}) = \vec{u}^T \bold{D} \vec{u}$. Using this expression for $D$ and the definition of $b$ from before, we can express the diffusion signal as such.
    
\begin{equation}
S(\vec{u}) = S_0 e^{-b (\vec{u}^T \bold{D} \vec{u})}
\end{equation}

This equation can be written in a linear form by taking the log of both sides and moving some terms.
\begin{equation}
\label{eqn:linearFormDti}
\log{\frac{S(\vec{u})}{S_0}} = -b (\vec{u}^T \bold{D} \vec{u})
\end{equation}

The elements of the diffusion tensor can be estimated by imaging with a series of non colinear gradient directions. Because the tensor is known to be real valued and positive semidefinite, there are 6 unique elements in the tensor. The unique elements of the tensor can be estimated several different ways. Equation \ref{eqn:linearFormDti} has a linear form and, when enough signals are available, $\bold{D}$ can be solved for by inverting the system of equations in a way that minimizes the residual error. This approach is known as the linear least squares (LLS) method for estimating diffusion tensors from diffusion MRI signals. However, the LLS method assumes that the error terms, or noise, in the system of equations is independent and identically distributed (iid). It is true that the MRI noise, across gradient directions, can be modeled as iid under some assumptions. However, the linear form of the equation models the log of the signal, so the error terms are in fact not iid at all. To address this issue, the weighted least squares (WLLS) approach is often used to fit estimate the diffusion tensor \cite{Jones_2010}. While the WLLS approach partially addresses the varying noise distribution, more advanced methods have also been proposed to estimate the diffusion tensor. For example the RESTORE method iteratively fits the diffusion signal, at each iteration identifying outlier signal measurements \cite{Chang_2005}. While this process allows RESTORE to be robust to patient motion and certain artifacts in the diffusion signal, it comes at the cost of additional computational complexity. The strength of the diffusion tensor comes from the simplicity of its model. This simple model allows us to describe powerful metrics such as mean, axial and radial diffusivity, as well as the seemingly ubiquitous fractional anisotropy. However this strength is also the weakness of this model. The simplicity of the model means that it cannot model the crossing of white matter tracts or multiple water compartments with different diffusivity values. To address these limitations, more complex models have been built to describe the diffusivity profile measured by diffusion MRI.