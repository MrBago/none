Even though the SH noise estimation performs well in simulation, it is
important to consider the impact of the SH order on the quality of the
noise estimate. The noise estimate is a sample statistic that is
computed using a finite sample from the noise distribution. Assuming the
noise distribution is a standard normal distribution, the expected
variance of a noise amplitude estimate computed with\(\text{\ d}\)
degrees of freedom is \(\sqrt{\frac{2}{d}}\). This is the stochastic
component of the RMS error, or the RMS error one would expect purely
from random sampling effects (Figure \ref{fig:simGraph} red line). Because the number of
degrees of freedom in the SH noise estimate go down as a function of SH
order, we expect the RMS error to rise with SH order. Looking closely at
Figure \ref{fig:simGraph}, one can see that the RMS error in the SH estimates (Figure \ref{fig:simGraph}
green lines) closely match the analytically predicted stochastic RMS
error (Figure \ref{fig:simGraph} red line) begging with SH order 4. Where the two lines
match, the only source of error in the SH estimate is sampling. At SH
order 2, the model does not fit the data well and produces an
overestimate for the noise amplitude and a high RMS error. This effect
is larger at higher b-value because the signal is composed of more high
order SH content. In order to minimize RMS error in the noise estimate a
SH order must be chosen to balance accurate model fitting with
preserving more degrees of freedom for the noise estimate.
