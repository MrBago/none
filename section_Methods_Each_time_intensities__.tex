\section{Methods}

Each time intensities of an image are measured, a noise sample is drawn
from a particular, unknown noise distribution. For the purposes of this
study, we attempt to measure the variance of this distribution, termed
``noise amplitude'', for each voxel. When an MRI sequence is repeated,
the noise amplitude can be estimated directly from these repeated
acquisitions. This approach, however, is not practical for either
clinical or research applications; if there is extra time in a study for
dMRI, the addition of gradient directions or b-values is usually
preferred over repeating the same sequence. This study, therefore,
focuses on methods that are able to estimate the noise amplitude from a
single typical HARDI acquisition. This section describes the four
specific methods tested in this study. The noise estimation methods
described in this study will be made available as part of Diffusion
Imaging in Python (Dipy), an open source software library \cite{Garyfallidis_2014}.

\subsection{Theory}
\subsubsection{B0 noise estimate}
The first method we evaluated is the simplest and does not use a
bootstrap procedure. Repeating an entire MRI study is impractical, but
it is common practice to acquire multiple minimally diffusion weighted
images (b0 images) in a typical dMRI sequence. If multiple b0 images are
available, a noise map can be estimated from the observed signal
variance across these images. In this paper, we refer to noise amplitude
estimated in this manner as the b0 noise estimate.

\subsubsection{Tensor noise estimate}
The second and third methods we evaluated are both residual bootstrap
methods, but employ different models. Residual bootstrap methods have
been used to estimate uncertainties in metrics derived from diffusion \cite{Chung_2006, Haroon_2009} and to inform fiber tracking \cite{17911030}. These methods work by generating bootstrap data sets, which are
analogous to simulated data acquisitions. These bootstrap data sets can
then be used to make inferences about the distribution from which the
data was sampled. In this study, we applied these methods to infer the
underlying variance of the diffusion imaging noise distribution.

One model we used for generating bootstrap data sets was the diffusion
tensor. Our implementation closely follows the description given by
Chung et. al. and is briefly described here. The diffusion signal is
modeled as

\begin{equation}
S(g_i) = S_0e^{-bg_i^TDg_i}
\end{equation}

where $S(g_i)$ is the diffusion signal measured with gradient $g_i$, $S_0$ is the signal
with no diffusion weighting, $b$ is the diffusion weighting, $D$ and is the diffusion tensor. We use this model to generate bootstrap data sets
using the following method. The tensor equation can be re-written in the
common linear regression form as

\begin{equation}
y = X\beta + \epsilon
\end{equation}

where $y=[ln(\frac{S(g_2)}{S_0}), ln(\frac{S(g_1)}{S_0}), ..., ln(\frac{S(g_N)}{S_0})]$ is the log of the normalized measured signals, $X$ is the design
matrix derived from the gradient table, and $\beta$ is the parameter vector to
be fit (composed of the 6 unique elements of the diffusion tensor), and $\epsilon = [\epsilon_1, \epsilon_2, ..., \epsilon_N]$
is the set of model residuals. It has been shown that weighted linear
least squares (WLLS) outperforms ordinary least squares (OLS) for
fitting the diffusion tensor model \cite{Veraart_2013}. The weighted least squares estimate of $\beta$ is

\begin{equation}
\hat{\beta} = (X^TWX)^{-1}X^TWy
\end{equation}

where $W$ is a diagonal matrix of weights such that

\begin{equation}
W_{ii} = \hat{S}_i^2
\end{equation}

\begin{equation}
\hat{S}_i = S_0e^{\hat{\mu}^{OLS}_i}
\end{equation}

\begin{equation}
\hat{\mu}^{OLS} = Hy
\end{equation}

\begin{equation}
H = (X^TX)^{-1}XT
\end{equation}

The matrix $H$ is often called the hat matrix. In order to use this fit to
generate bootstrap data sets, a set of centered, leverage-corrected
residuals must be computed. The raw residuals are given as

\begin{equation}
e = y - X\hat{\beta}
\end{equation}

The centered, leverage-corrected residuals $q$ are given as

\begin{equation}
r_i = \frac{e_i}{w_i^{-1/2}(1-h_i)^{1/2}}
\end{equation}

\begin{equation}
q_i = r_i - \bar{r}
\end{equation}

where $w_i$ and $h_i$ are the diagonal elements of the $W$ and $H$ matrices, respectively. A bootstrap data set, $y^*$, is produced by randomly drawing values from $q$ for $\epsilon^*$ (with replacement). This is given by

\begin{equation}
y^*_i = x_i\hat{\beta} + w_j^{-1/2}\epsilon^*
\end{equation}

\begin{equation}
S^*(g_i) = S_0e^{y_i^*}
\end{equation}

Once bootstrap datasets have been generated from the diffusion data set,
the noise amplitude at each voxel can be estimated by taking the
variance of the bootstrap data sets for that voxel. In this paper, we
refer to noise amplitude estimated in this manner as the DTI noise
estimate.