

\section{Introduction}

	The negative health effects of being overweight or obese are well known and include higher risk of cardiovascular disease, diabetes and even some cancers. The neurological and cognitive effects of having high body weight are less well known. Obesity has been associated with both lower levels of cognitive function and an increased risk of dementia later in life. Recent work has found that obesity and high body mass index (BMI) are associated with declines in several cognitive domains including executive functioning, attention and motor skills \cite{Memel_2016,Yau_2012,Gunstad_2007,Nguyen_2014}. Additionally, persons with high BMI in mid-life have a higher risk of developing both Alzheimer's Disease and non-Alzheimers dementia. Even children with higher BMI appear to have declined cognitive abilities compared to their normal weight peers \cite{Liang_2013}, and among college athletes with excellent cardiovascular health, high BMI is associated with lower cognitive function \cite{Fedor_2013}. However, there is also evidence that higher body weight may play a neuro-protective role later in life \cite{Hsu_2015}. Brain imaging can help us better understand the mechanisms that underlie these links between body weight and cognition.
	
	In addition to cognitive effects of body weight, recent work has demonstrated that high body weight can effect the macro and micro-structure of the brain. Lower brain volumes have been reported in people with high BMI \cite{Ho_2010a,Raji_2009}. Additional work focused on regional differences in brain volumes has found both gray matter (GM) and white matter (WM) atrophy, particularly in the temporal and parietal regions of the brain \cite{Willette_2015}. Micro-structural alterations in brain tissues associated with high BMI and obesity have also been reported. Several studies using diffusion tensor imaging (DTI) have found lower fractional anisotropy (FA) and possibly lower axonal coherence in several white matter tracts, including the corpus callosum and cingulate bundle \cite{Bettcher_2013}. These DTI findings are supported by a spectroscopic imaging study which found that high BMI correlated with lower concentrations of N-acetylaspartate, a marker of neuronal viability \cite{Gazdzinski_2008}. While these findings demonstrate that the observed neuro-cognitive differences between persons with high BMI and normal weight are mediated through alterations in the brain, there is disagreement in the literature about which brain regions are implicated and to what extent.
	
	In this work we apply a new methodology for examining the associations between white matter biomarkers and patient outcomes to the a large, public data set of young subject in order to better understand the link between body mass and brain health. This new methodology maps white matter tissue markers onto a large, intersecting set of white matter pathways reconstructed using advanced tractography methods. The tract based markers are then statistically modeled to identify associations between white matter structures and health outcomes. We present the results of our white matter analysis along with cortical thickness analysis and show that the BMI effects to the brain, in this young population, are most measurable in the white matter tissues.
    
\section{Methods}
	For this study we used the S900 release of HCP \cite{Van_Essen_2013}. This release included 766 subjects with complete diffusion MRI data sets and BMI information. The mean age of the group was 28.8 +- 3.7 years old and the group had 426 females and 339 males. The sex of 1 subject was not disclosed. The group had a mean BMI of 26.4 +- 5.1. The HCP acquires 3 similar diffusion MRI scans with 1.25mm isotropic resolution. Each of the 3 scans have different gradient directions so they can be combined into one data set with 270 unique gradient directions. These 270 directions are distributed over 3 shells, with b-values of 1000, 2000 and 3000. The combined data set also contains 18 volumes with minimal diffusion weighting (b0 images). Both unprocessed and preprocessed versions of the diffusion MRI images are available publicly. For this study we used the preprocessed images. These images were reconstructed using a SENSE1 MRI image reconstruction and then eddy corrected using a gaussian process predictor \cite{Sotiropoulos_2013}. In addition to diffusion MRI images, we also used the structural T1 images, the MNI registrations, and the freesurfer segmentations of those T1 images provided as part of the structural preprocessed HCP data set \cite{Glasser_2013}. We chose the HCP data for this study because it contains a large number of patients, high resolution structural imaging, freesurfer processing, and high resolution and high quality diffusion MRI images. This data is also publicly available for repetition and follow up studies.
	
\subsection{Fiber Tracking and DIffusion Metrics}
    From the diffusion MRI data we extracted DTI metrics for a large number of white matter pathways identified using whole brain fiber tracking. These pathway specific metrics served as features in our statistical analysis. In this study we focused on two metrics specifically, fractional anisotropy (FA) and mean diffusivity (MD). Our goal was to find the mean FA and mean MD along as many white matter pathways as possible. To achieve this goal we used whole brain tractography to create a tractogram for each subject using fiber tracking tools in the Dipy software library \cite{Garyfallidis_2014}. Our fiber tracking methodology is described below. 
We used probabilistic fiber tracking with a step size of .5 mm to generate tractograms for each subject. We created gray matter and white matter masks using the freesurfer segmentation results provided in the HCP data set. The probabilistic fiber tracking was seeded in the white matter voxels adjacent to gray matter tissues using 27 seeds per voxel. The probabilistic fiber tracking was restricted using the white matter as a binary tissue classifier. Streamlines were included in the tractogram only if both endpoints of the streamline were in the gray matter mask. We used a Multi-Tissue Constrained Deconvolution model, with spherical harmonic order 8, to estimate an fiber orientation distribution (FOD) at each voxel \cite{Jeurissen_2014}. These FODs were used, with a 30 degree angular threshold, to generate probabilistic tracking directions.

    Using the whole brain tractograms, we identified the white matter pathways that were most promising for further analysis. For this study, we estimate a white matter pathway to be a path along all the streamlines that connect two gray matter regions. Using cortical and subcortical gray matter regions from the freesurfer segmentations, we identified 1102 such white matter pathways which were identified in all the subjects of our study. For each white matter pathway, we measured the average MD and FA to produce 2204 total measurements that we then used as features in our statistical analysis.
    
\subsection{Statistics}
    For statistical analysis we used scipy and scickits learn \cite{jones2001open,Hill,Nelli_2015}. Tissue volumes, cortical thickness, age, gender and BMI information were obtained from the HCP data set. We used a partial least squares (PLS) regression \cite{leguina2015primer} to model the association between white matter measures, cortical thicknesses and BMI. In order to validate our results, we split the original dataset into two groups. The training group consisted of all 437 subjects released by the HCP in the S500 release or before. We set aside 329 subjects from the S900 release to be our test set. To determine the optimal number of latent variables for the PLS regression, we used a random 10 fold cross validation with 1000 iterations and fit models with up to 12 latent variables. Once the number of latent variables was established, we fit a PLS regression using all subjects from the training set and used the test set to validate the result. The results of the PLS regression were expressed as an image, denoted here as the effect map. See Appendix 1 for more on how the effect map was created. Using all the subjects in our training set, we warped the effect for each subject to MNI space and averaged the effect maps to produce a normalized, average effect map.
