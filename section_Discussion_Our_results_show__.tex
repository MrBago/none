\section{Discussion}

Our results show that using a SH model to estimate noise maps for HARDI
data is more accurate and more precise than using a DTI model or using
the currently available b0 and non-local means methods. The accuracy of
the noise map is mostly independent of the SH order used, provided that
a sufficiently large order is used. However, the precision of the
estimate does depend on the choice of SH order. Even though the SH model
noise estimate is better than the tensor or B0 estimates, we did observe
some discrepancy between SH noise estimate and the ground truth noise
measured by repeated data acquisition, not all of which can be explained
by stochastic variation. The simulation we used to estimate the expected
stochastic RMS error is conservative and we would not expect to achieve
that level of precision, even under ideal conditions. However, we
believe the precision of the noise maps may be improved if better eddy
current correction and image alignment methods are used in the
processing of the raw data. The repeated gradient acquisition data we
used for our ground truth noise measurement has very consistent
alignment. The consistent alignment is not surprising because each of
the repeated images is impacted by the same eddy current distortions. On
the other hand, the data used to produce the bootstrap noise estimates
has visibly worse alignment between the images. The areas where
misregistration is most pronounced are consistent with the areas of
largest differences between the noise estimate and the ground truth.
While we explored possible parameters for improving the alignment, the
simple eddy current correction method we used is not capable of fully
correcting the eddy current distortions present in this kind of data.

Other factors may also contribute to differences between the ground
truth noise measurement and the bootstrap noise estimates, but it is
important to note that there is no evidence that model failure is one of
these factors when sufficiently high SH order is used. This is in
contrast to the DTI method, where there is strong evidence for model
failure. In the presence of model failure, the bootstrap procedure
overestimates the residuals and overestimates the variance in the metric
of interest. In this case, we would expect a model failure to result in
a positive bias, or an overestimate, of the noise amplitude. The DTI
results show an overestimation of the median noise. The noise maps show
that this overestimation is not uniform over the image, but most
predominant in the white matter regions where we expect crossing voxels.
Additionally, the SH model produces a lower noise estimate in these
regions, which is consistent with the expectation that the SH model can
better represent crossing voxels. The SH model shows evidence of model
failure at order 2 and order 4 for b=3000. However, there is no evidence
of model failure at higher SH orders. In the presence of model failure,
we expect the median noise estimate and RMS error to decrease as higher
SH order models decrease the model failure.

In this study, we have focused on noise estimation for single shell
q-ball data, however these methods could still be useful for studies
that acquired multi-shell diffusion acquisitions. The noise in each
shell could be analyzed independently using these methods as long as
there were at least about 20 gradients acquired in each shell. This
approach is the simplest and would work for acquisitions where imaging
parameters are adjusted independently for each shell to maximize SNR at
each shell. The disadvantage of treating each shell independently is
that model parameters must be fit resulting in a noise estimate with
fewer degrees of freedom. To avoid this one could use one of the several
recently published methods that use SH functions to fit multi-shell
data \cite{Jeurissen_2014}. Because
these methods make additional assumptions about the nature of the data
being modeled, they would need to be validated independently, but that
work is outside the scope of this paper.

Understanding which combination of model and parameters produces the
best voxelwise noise map is important not only for accurately estimating
noise, but also for choosing the right bootstrap procedure in other
applications. In recent years, many new models for characterizing
diffusion profiles in brain tissue have been proposed. Many of these
models, such as NODDI\cite{Zhang_2012} or CHARMED\cite{Assaf_2005},
require complex non-linear fitting of the diffusion data. Residual
bootstrap methods could be used to estimate the standard errors produced
by these models, but bootstrapping the models directly is not practical
because leverage correcting the residuals produced by these models is
impractical or impossible. However, one could use a different model to
produce bootstrap datasets and then apply the model of interest to those
datasets in order to estimate standard errors in the model parameters or
metrics. The quality of an estimate produced using a bootstrap procedure
depends on how well that procedure can generate bootstrap samples that
are representative of the distribution from which the data is sampled.
Any bootstrap sample that represents the underlying data distribution
should, at minimum, be able to estimate the variance (or noise) of that
distribution. Therefore, when picking a bootstrap procedure for any dMRI
application, one should consider how well that procedure estimates the
imaging noise.

While this study demonstrates that noise maps can be estimated from dMRI
acquisitions, how well the methods presented here perform may depend
somewhat on the MRI acquisition sequence used and the biological
characteristics of the tissue being imaged. Two important criteria are
required for model residual bootstrap methods to be effective. First,
the model must be a reasonably accurate representation of the diffusion
characteristics of the biological tissue. Our results indicate that the
tensor model fails at high b-values, causing the noise to be
overestimated. While the SH model has similar accuracy over a b-value
dependent range of orders, the lowest order in that range provides best
precision in the noise estimates. The second criterion for accurate
noise estimation is that the imaging noise must be approximately
normally distributed. MRI imaging noise is generally believed to have a
Rician distribution that can be approximated as normal when the SNR is
sufficiently high. Therefore this criterion is satisfied for most dMRI
data sets. However, data sets that approach the noise floor will result
in some bias in the noise estimates due to violation of the normally
distributed noise assumption.