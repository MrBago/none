\chapter{Introduction}

Diffusion weighted magnetic resonance imaging (MRI) is an imaging technique that has allowed unique insights into both the microstructural properties and the organization of cerebral white matter tissues. This technique measures the relative displacement of water molecules in the biological tissues and is highly sensitive to any microstructure which restricts this diffusion. Diffusion MRI is particularly useful in tissues with a high degree of organization, such as the white matter tissues of the brain. This organization means that the restricted diffusion is coherent and measuring the diffusion signal allows us to not only infer the presence or absence of diffusion restricting elements, but also how they're organized and other tissue properties. For example, in the white matter of the brain, the diffusion characteristics of the tissue can give us insight into the organization of axons, the level of myelination of those axons and possible pathology in these tissues. The level of information contained in diffusion imaging makes this imaging technique both powerful and challenging to work with.
	
Diffusion images cannot generally be read by human specialists, but instead must be modeled using computational techniques. These modeling techniques can produce either composite images or 3d renderings, which can then be used in clinic or for research, or quantitative measurements which can then be used as biomarkers of brain heath or disease progression. However, in order to maximize the utility of diffusion imaging one needs to pick the right imaging protocol and modeling approach for a given problem. The modeling of the diffusion signal can be approached both as a local problem, modeling the characteristics of a given brain region or voxel using the diffusion signal specific to that structure. The model can also be thought of as a whole brain model, trying not only the estimate local tissue properties, but also the organization of connections and pathways that contribute to the architecture of the whole brain. Fiber tracking, or tractography, describes the process of using local directional information from brain tissues to build reconstruct these tracts and pathways. These techniques are fairly new and actively being developed and have had some success in describing the organizational properties of the human brain, or the human connectome.

In this work, I present some common diffusion modeling techniques that have been applied to diffusion MRI, focused specifically on modeling techniques that estimate directional information which can be used for fiber tacking. I present and compare several methods for estimating diffusion MRI noise and in the process discuss how noise estimates for diffusion MRI acquisitions can help identify model failures inform the choice of model for an acquisition type. I further present a framework for thinking about and implement fiber tract reconstruction from diffusion MRI data. Finally I present an application of these methods to a large, public data set for the purpose of understanding the impact of high body mass index on brain health. 
 %