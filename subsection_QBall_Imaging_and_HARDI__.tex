\subsection{QBall Imaging and HARDI}
    High angular resolution diffusion imaging (HARDI) extends the capabilities of DTI by providing more detailed information about the diffusion profile of water molecules then can be described by a tensor model \cite{Tuch_2002}. While there is no strict definition of what qualifies as a HARDI scan protocol, a typical HARDI protocol will consist of many gradient directions, more than 50, acquired at a relatively high b value, 2000$s/mm^2$ or more. The HARDI protocol is better, relative to DTI, and able to capture more complex microstructural tissue architecture not only because of the more complete directional coverage, but also because the higher b-value makes the diffusion MRI scans more sensitive to the diffusion of water molecules at a shorter length scale. HARDI protocols try and maximize angular information relative to scan time, however this can come at the cost of additional information that might be gained by imaging at multiple b values. Even with that limitation, HARDI scan protocols represent an efficient balance between gathering complete information and minimizing scan times. They are especially useful for fiber tracking applications where directional information is most essential. Data collected using a HARDI scan protocol can be modeled using several methods, including using a diffusion tensor. However, more complex modeling methods such as using a q-ball reconstruction allow us to fully exploit this type of data.
    
    The goal of the q-ball reconstruction is to estimate the diffusion orientation distribution function (ODF) from the diffusion signals. To fully understand the ODF one must first consider the diffusion probability density function (PDF), the density of water molecules as a function of their 3D displacement. This diffusion PDF is the most complete representation of the diffusion profile in an imaging voxel and it turns out that the diffusion MRI signal is actually the 3d fourier transform of the PDF \cite{Wedeen_2005, Wedeen_2008}. That is $S(\vec{q}) = \bold{F}[P(\vec{r})]$ where $\vec{q}$ is defined as $\vec{q} = \gamma\delta\vec{g}/(2\pi)$ and $\vec{g}$, $\gamma$, and $\delta$ are the diffusion gradient, the gyromagnetic constant and gradient duration \cite{Tuch_2004}. Estimating the PDF directly from the diffusion signal by measuring a full diffusion spectrum in q-space is called diffusion spectrum imaging (DSI). In HARDI imaging however, only one b value is used so only one shell, or "ball", of the diffusion spectrum is collected. This prevents the reconstruction of the full PDF, but the ODF can still be reconstructed from HARDI data.
    
    The diffusion ODF is the radial integral of the diffusion PDF, however this integral has been defined two different ways in the literature. These different definitions of the ODF give slightly different results in the reconstruction, but they both aim to measure the net diffusion of water molecules, as a function of radial direction. The first definition was used by Wedeen et al. when estimating fiber directions using the DSI approach and then later used by Aganj et al. and Tristán-Vega et al. to define a q-ball reconstruction where the radial integral is expressed using a constant solid angle \cite{Aganj_2010, Trist_n_Vega_2009}. That is, the ODF is defined as $\Phi(\vec{u}) = \int{P(\vec{r})r^2dr}$ where $\vec{r} = r\vec{u}$ and $\vec{u}$ is a unit vector. The alternate definition of the ODF, $\Psi = \int{P(\vec{r})dr}$ was used by Tuch in the original formulation of the q-ball reconstruction \cite{Tuch_2004}. In both cases, the goal is to represent the directional component of the diffusion profile while collapsing the radial information.
    
    The key insight in q-ball imaging is that the radial integral of the diffusion PDF can be expressed as a great circle integral of the diffusion signal, or some function of the diffusion signal \cite{Tuch_2003, Aganj_2010}. In reality, this relationship is a simplification, and each of the q-ball variants uses a different set of assumptions in order to reduce what should be a plane integral into a great circle integral, but ultimately expresses the diffusion ODF as a great circle integral over some function of the diffusion signal. Equation \ref{eqn:funkRadon} gives this relationship, known as the Funk radon transform.

\begin{equation}
\label{eqn:funkRadon}
F[f(\vec{w})](\vec{u}) = \int\int_{\vec{u}_{\perp}}{f(\vec{w})\delta(|w|-1)d^2\vec{w}}
\end{equation}

Under this formulation, both the diffusion signal and the ODF are expressed as functions defined on $\bold{S}^2$ in $R^3$.

    The spherical harmonic (SH) series represents a set of orthonormal basis functions defined on $\bold{S}^2$, or the surface of a unit sphere. This series is analogous to the fourier series on the surface in that a fourier decomposition of a function is essentially a frequency representation of the function and any continuous complex function defined on $\bold{S}^2$ can be expressed as a sum over this series. Figure \ref{fig:SphHarm} shows a visual representation of first several SH functions.
    
    
    
   

%end