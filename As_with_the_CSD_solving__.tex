As with the CSD, solving the multi-tissue deconvolution problem requires a constrained inversion of the convolution matrix. Jeurissen et al. propose using a convex QP solver in place of the iterative method proposed by Tournier et al \cite{Mehrotra_1992}. This convex optimization technique is designed to find the optimal solution to the problem $argmin||X\vec{c} - \vec{s}||^2$ under the constraint that $A\vec{c} >= 0$. We can formulate the MT-CSD model in this way so that $X$ is matrix such that it maps the coefficients of the model, \vec{c}, to expected diffusion signals for each gradient used to acquire the observed signal \vec{s}. We can also build a matrix $A$ which samples the FOD associated with each tissue class on a unit sphere. One advantage of using a QP solver is that it guarantees positive volume fractions for each tissue type of the model.

Because the MT-CSD estimates the diffusion signal as having contributions from different tissue types, we can estimate volume fractions from the model fit. Volume fractions are estimated using the model coefficients associated with the 0-degree SH term for each of the tissue types. These volume fractions are scaled by the response function amplitudes and are not guaranteed to sum to one, so one must take care when using these values. However, the volume fractions still serve as a measure of the relative contribution of each tissue type to the measured diffusion signal from different regions of the brain. Figure \ref{fig:volFrac} shows the volume fraction map computed using MT-CSD with three tissue types, CSF as red, gray matter as green and white matter as blue.