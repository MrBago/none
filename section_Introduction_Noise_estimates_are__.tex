\section{Introduction}

Noise estimates are widely used in image processing tools to evaluate
image quality and to estimate signal to noise ratio (SNR). Advanced
analysis methods use noise estimates to inform data-driven inferences.
The robust model fitting method RESTORE, for example, uses a noise
estimate to reject outlier data during the model fitting process. Noise
in an MRI image is typically estimated using the signal variance in a
region that is expected to have uniform signal. These types of simple
noise estimation methods fail in the presence of modern imaging and
reconstruction methods. More advanced methods have been proposed to
measure noise in images acquired using parallel imaging, but these
methods have their own limitations. For example, two recent methods
utilize the raw data to estimate the noise \cite{Robson_2008, Kellman_2005}, but because these methods work with raw data, they must be implemented as part of the
vendors' reconstruction algorithm or the raw data must be saved for post
hoc analysis. Other noise estimation methods have been proposed which
overcome these limitations by analyzing the statistical properties of
the intensities in the image background \cite{19346143, 27845653}. These
background methods rely on the known statistical properties of
background voxels, voxels made up primarily of air. These methods
simultaneously identify which voxels belong to the background and
estimate the noise amplitude. One major limitation of these background
methods is that they greatly underestimate the noise if a reconstruction
filter is used or if other post processing steps change the statistical
distribution of the background voxels\cite{Dietrich_2007}. In general,
past methods were not developed specifically for dMRI and each of the
methods has limitations that must be considered before its use in dMRI
applications.

In this work, we present and evaluate several methods for estimating
spatially varying noise in high angular resolution diffusion imaging
(HARDI) data sets. To our knowledge, this work demonstrates the first
method to estimate voxelwise noise maps for dMRI data sets. We focus on
HARDI scans because they can be completed in clinically practice
acquisitions time and they also support the use of many standard and
advanced diffusion models, for example DTI, Q-ball \cite{Tuch_2004}
 or constrained spherical deconvolution (CSD) \cite{Tournier_2007}. Because the
methods we present estimate noise maps directly from signal regions in
the image, they produce detailed noise maps in addition to an overall
noise estimate. In this paper, we compare the noise maps produced by each
of the four methods to simulated and measured noise maps to evaluate the
accuracy and precision of each of the four methods.