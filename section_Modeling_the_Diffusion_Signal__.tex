\section{Modeling the Diffusion Signal}
\subsection{The Diffusion Tensor}
    Diffusion Tensor Imaging (DTI) is among the oldest and most commonly used models for diffusion MRI applications. In order to fit a diffusion tensor model to the diffusion signal, a minimum of six well dispersed gradient directions are necessary \cite{Hasan_2001}. However, it is often better to acquire at least 20-30 gradient directions to improve the accuracy of the tensor estimate \cite{Jones_2004}. Even though it may seem that acquiring 30 directions may take significantly more scan time than acquiring 6 directions, in practice a number of excitations (NEX) average is used in DTI studies to improve signal to noise ratio (SNR). Therefore, more gradient directions can be acquired in the same scan time by reducing the NEX value. The increased number of acquired gradient directions compensates for the reduced SNR of each individual image and allows for a more accurate tensor fit. Another consideration when designing an acquisition sequence for DTI is the b-value or values to use. The diffusion tensor best fits data collected at b-values less than 1500$\frac{s}{m^2}$, \cite{Clark_2000}. This following chapter will also discuss the limitations of the tensor model at higher b-values. Some studies have suggested that using multiple b-values can improve the reliability and reproducibility of DTI metrics \cite{Correia_2009}. However, DTI studies often use a set of well dispersed diffusion gradients with a constant b-value in addition to one or more non-diffusion weighted images. One advantage of the DTI model is that it can be used to fit most diffusion MRI acquisitions, even sequences optimized for fitting other models tend to produce accurate and reproducible results.