\subsubsection{SH noise estimate}
The third method for estimating noise amplitudes evaluated in this study
is the same as the above method, except the spherical harmonic (SH)
functions are substituted for the tensor as the diffusion model. The SH
functions can be used as a frequency space representation of functions
defined on the sphere, analogous to the Fourier series for 1D
functions \cite{Hess_2006}. The SH functions form an orthonormal basis of
functions on the sphere, thus any complex-valued function on the sphere
can be represented as a sum of SH functions as follows

\begin{equation}
y(g_i) = \sum_{l=0}^{\inf} \sum_{m=-l}^l{c_{l,m}Y_{l,m}(g_i)}
\end{equation}

where $c_{l,m}$ are the coefficients associated with each of the SH functions $Y_{l,m}$ of degree $l$ and order $m$, and $g_i$ is a unit vector. Even though the SH functions
are not a diffusion model, when truncated at a maximum SH order, they
can be used as a model for residual bootstrap purposes.

The diffusion MRI signal has two properties that can be exploited to
refine this model. First, the signal is known to exhibit antipodal
symmetry, thus does not have any contribution from the antipodally
asymmetric SH functions (those with odd degree) \cite{17911030}.
Second, the signal is known to be real-valued. These signal properties
allow us to use a set of real valued SH functions with even degree.
Because there are even degree SH functions with degree less than or
equal to, we can define a set of real valued, even SH functions where

\begin{equation}
X
\end{equation}

The diffusion signal expressed in terms of a truncated SH series becomes:

\begin{equation}
y(g_i) = \sum_{j=1}^N{c_jY_j(g_i) + e_i}
\end{equation}

or

\begin{equation}
y = Xc + e
\end{equation}

where the matrix $X$ is defined $X_{ij} = Y_j(g_i)$ , $c = \{c_1, c_2, ..., c_N\}$ and is a vector of coefficients. These
coefficients can be efficiently estimated using a least squares method
and take the form $c = (X^TX)^{-1}X^Ty$.

The bootstrap samples are created using leverage-corrected and centered
residuals. The raw residuals can be computed as

\begin{equation}
e = y - Xc
\end{equation}
\begin{equation}
e = y - X(X^TX)^{-1}X^Ty
\end{equation}
\begin{equation}
e = y - Hy = (I - H)y
\end{equation}

where $H = X(X^TX)^(-1)X^T$ and $I$ is the identity matrix. The SH fitting is linear, so the
leverage correction term can be computed directly from $H$. The
leverage-corrected residuals are

\begin{equation}
r_i = \frac{e_i}{\sqrt{1 - h_i}}
\end{equation}

where $h_i = H_{ii}$ are the diagonal entries of the hat matrix. The leverage-corrected residual vector $r$ can also be written as

\begin{equation}
r = L(I - H)y
\end{equation}

where $L$ is a diagonal matrix such that $L_{ii} = 1 / \sqrt(1 - h_i)$. The centered residuals are defined as $q_i = r_i - \hat{r}$ or equivalently

\begin{equation}
q = C_NL(I - H)y
\end{equation}

where the centering matrix is defined as $C_N = 1 - \frac{1}{N}\vec{1}\vec{1}^T$. $q^*$ is created by sampling from $q$
with replacement and the bootstrap data sets is defined as

$S^* = Xc + q^*$

The noise amplitude of the diffusion signal can be estimated by taking
the variance of these bootstrap samples. We refer to the noise amplitude
estimated in this way as the SH noise estimate. In practice, we don't
need to generate bootstrap samples $S*$ if we notice that $E[var(S*)] = \frac{1}{N - 1}var(q)$. Also, the matrix $C_NL(I - H)$
can be computed a-priori, so the computation of the SH noise estimate in
each voxel is reduced to two operations, a matrix multiplication and the
computation of variance.

\subsubsection{Non-local means noise estimate}
The fourth method we evaluated for estimating noise maps was the
application of non-local means filter\cite{Coupe_2008}. This
de-noising filter has proven to be a valuable tool for improving data
quality of dMRI images before diffusion models are fit to the data or
fiber tracking is performed. By taking the difference between the noisy
data and the denoised data, we compute a set of residuals, which can be
used to generate a voxelwise noise map. Because non-local means
filtering requires the average noise amplitude as an input, it needs to
be used along with another noise estimation method. For this study, we
were only interested in the quality of the noise maps, so we used the
known noise amplitude measured from the ground truth data set as the
input. This procedure gives an upper bound, best possible result, for
the non-local means noise estimation method because in practice the
average noise must be computed from the data and may have some error. In
order to produce noise maps from the non-local means residuals, the
residuals must be scaled so that the average noise amplitude matches the
expected value. This step replaces the leverage correction step of the
other methods \cite{davison1997bootstrap}.
After scaling, the non-local means noise maps can be compared directly
to the ground truth noise maps.