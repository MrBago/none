\section{Discussion}

Our study did not find an association between lower brain volume and higher BMI values in these young adults. Previous work in older populations has found that BMI is associated with reduced GM in many brain regions including the parietal, frontal occipital, and temporal lobes \cite{Ho_2010,Raji_2009}. However, in this study we only observed cortical thinning in three temporal GM regions. This apparent difference may be explained by the fact that the HCP subjects are much younger than the populations in these previous studies. This difference would arise if the reduced GM tissue observed in these studies of older individuals is due to increased aging related atrophy in persons with higher BMI. The aging hypothesis is consistent with previous work which found that dietary changes in young, obese mice can reverse obesity related cognitive declines \cite{Sims_Robinson_2016}. Previous work has also shown that rats fed a high fat diet have higher levels pro-inflammatory cytokine expression in several brain regions \cite{Boitard_2014}. An inflammatory mechanism may be the reason that we observed a significant number of cortical GM regions which exhibit a positive association between cortical thickness and BMI. If both inflammation and atrophy contribute to the observed GM tissue thicknesses and volumes, the net effect may be small, at least in younger patients.

The strongest association we found in this population between imaging metrics and BMI was in the WM. Using a PLS model, the whole brain tract based diffusion metrics explain about 20\% of the observed variance of BMI in this population. By contrast, cortical GM thicknesses and age explain less than 5\% of that variance. Using our tractography approach, we modeled the white matter of the brain as a collection of a large number of overlapping WM pathways, both short range and long range, that when taken together constitute the connectivity of the brain. These pathways consist of both large bundles that mainly overlap with well known WM structures, such as the corpus callosum or superior lateral fasciculus, as well as smaller bundles that only minimally overlap with these well studied structures. This approach allowed us to organize regional WM information in a biologically motivated, data driven way without over simplifying the WM anatomy of the human brain. While tractography can be very noisy, the HCP data set has high resolution, multi shell diffusion MRI images with 270 unique gradient directions. The quality of this data, when combined with advanced diffusion modeling techniques, allowed us to create a robust and detailed model of the WM connectivity for each subject. Because the pathways of this connectivity model overlap, each voxel of the image can contribute to multiple pathways. When we used PLS regression to associate these pathways with BMI, we did not identify individual pathways that were specifically predictive of BMI, however when we computed effect maps from the PLS regression, we did find WM regions that contributed more strongly to the final BMI prediction.   


The largest contributions to the PLS model were from regions associated with the cingulate bundle and uncinate fasciculus. Both these tracts connect frontal and temporal regions of the brain. We found that the FA and MD values in the temporal aspects of both these tracts contributed negatively to the BMI prediction. Additionally, the FA and MD of the anterior aspect of the cingulate bundle and the FA of some frontal WM areas contribute positively to the BMI prediction. To understand this relationship further, we computed the weighted average FA and MD in the red (positive contributions) and the blue (negative contributes) regions of the effect maps. We found that the average MD in both areas, red and blue, had a negative correlation with BMI. Also, the average FA in the blue regions had a negative correlation with BMI, but the FA in the red regions had no significant correlation. These findings are consistent with the linear model of BMI, which found a negative correlation with both whole brain WM average FA and MD relative to BMI. Taken together, these results tell us that higher BMI is associated with lower FA and MD in much of the brain. This effect is strongest in the blue areas of the respective effect maps. However, higher FA and MD in the red regions of the respective effect maps is also predictive of higher BMI. This relationship exists either because the red regions are the least affected but higher BMI or because these red regions are the most correlated the blue regions in subjects with lower BMI and therefor serve as the best control regions for detecting BMI related differences in those blue regions. Both these explanations may account for some of the observed association in the model and the final result is that differences in FA and MD between the red and blue regions of the respective effect maps are more predictive of BMI than the FA and MD values taken from any given region of the brain.

Pervious studies have tried to identify WM effects related to BMI or obesity, but their results have been inconsistent. Our findings confirm some of the previously reported results, but may also help explain some of the inconsistency. Previous studies have found lower FA in some of the same WM pathways described in this paper. Specifically in studies of older subjects, one study reported lower FAs in the several parts of the corpus callosum, fornix, and cingulate bundle but did not find a relationship with FA in the uncinate fasiculus \cite{Bettcher_2013}. Another study, with similar methodology, considered these four WM tracts, but only found a relationship in the uncinate fasiculus, and no relationship between FA and BMI in the other three tracts \cite{Bolzenius_2015}. The relationship between FA and BMI has also been studied in younger subjects, but with smaller sample sizes. One such study found lower AD (consistent with lower FA and lower MD) in the superior lateral fasiculus and anterior thalamic radiations while another study found lower FA in the mid-brain, internal capsule and perihippocampul white matter \cite{Verstynen_2012,Kullmann_2016}. The discrepancy in these results may be due to a lack of sensitivity in the methodologies used for these studies. DTI metrics exhibit a large intersubject and regional variability within subjects. Differences in how ROIs are defined in different subjects and in different studies can also contribute to estimation errors. If BMI is associated with DTI metric changes in much of the brain, as we have observed in this study, then small associations with BMI can be observed in many different regions. However, these observations will be limited by the sensitivity of the methodology and sample size used. By modeling the whole brain WM based on a connectivity model, we were able to measure a stronger association with BMI than any previously reported association.
