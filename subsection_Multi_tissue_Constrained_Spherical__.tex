\subsection{Multi-tissue Constrained Spherical Deconvolution}

The multi-tissue constrained deconvolution (MCD) model of diffusion is a natural extension of the CSD diffusion model. Much like the CSD model, the MCD model attempts to reconstruct an FOD from the diffusion signal, however the MCD model allows for a more complex response function. Specifically the MCD approach models the diffusion signal as having contributions from multiple tissue types, each with an independent response function. In order to resolve contributions from multiple response functions, the MTD model needs a multi-shell diffusion acquisition protocol, as opposed to the single-shell HARDI acquisition used by the CSD model. In their initial presentation of this method, Jeurissen et al. use 3 tissue types in the MTD model: white matter, gray matter and cerebrospinal fluid (CSF) \cite{Jeurissen_2014}. They estimate a response function for each of the 3 tissues, however the gray matter and CSF response functions are assumed to be fully isotropic. A fully isotropic response function has no dependence on gradient direction, only on gradient amplitude. The white matter response function is estimated independently for each q-shell, following the same procedure as the CSD model. Figures \ref{fig:responseZView} and \ref{fig:responseXView} show the WM response function, computed from the WM tissues of a healthy subject, for a 3-shell, b of 0, 1000, 2000 & 3000, diffusion protocol.