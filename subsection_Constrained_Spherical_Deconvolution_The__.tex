\subsection{Constrained Spherical Deconvolution}

The constrained spherical deconvolution (CSD) model is similar to the q-ball model in that it attempts to estimate directional orientation information using data from a HARDI scan protocol. However, unlike the q-ball model, the CSD model doesn't attempt to estimate the ODF, instead it attempts to model something called the fiber orientation distribution (FOD). In q-ball imaging the water molecule is the basic unit under consideration. Each molecule contributes to the diffusion signal, and from the diffusion signal we attempt to model the net displacement of water molecules, at least in angular terms. The CSD model, however, considers the "fiber" to be the basic imaging unit. If we assume that each fiber contributes equally to the diffusion signal, and that there are no major contributions to the signal from other sources, we can reconstruct the configuration of fibers necessary to produce the observed signal. Biologically a fiber is roughly analogous to a small bundle of axons. If one could isolate such a bundle and image it in isolation, the diffusion signal it would produce would be what's known as the response function. In other words, the response function is the signal produced from a single imaging unit. If the tissue being imaged consists only of these fibers, then the measured signal is simply the sum of the response function from each fiber. The goal of the CSD model is to reconstruct the orientations of the underling fibers from the diffusion signal.

To build a CSD model, one needs to estimate the response function. Even though one can approximate the response function analytically, for example by assuming that the diffusion PDF associated with each fiber has a gaussian distribution, estimating the response function from data produces the best results in general. The simplest way to do this is to find an area of the an image known to have highly aligned axons, for example a region with high FA values. In this area of the image we can assume that all the fibers share the same orientation. If the fibers are all aligned, then the observed signal is simply a scaling of the response function. The signal from the voxels in this image area should be rotated so the fibers from all the voxels are aligned, we align the fibers with the z axis by convention, then the signals can be averaged to produce an estimate for the response function. This approach works reasonably well, however more advanced methods have also been proposed for estimating the response function of the CSD model. A recursive approach can be used to at once estimate the response function and determine which voxels can be treated as fully aligned fibers \cite{Tax_2014}. The idea is to refine the response function until the voxels that contribute to estimating the response converge.

With the response function computed, the CSD method models the signal as a sum over rotated response function. Each fiber in the tissue contributes the same response function to the signal, however the response function associated with each fiber is rotated to match the alignment of the fiber. The diffusion signal is the sum of these rotated response functions. In this formulation the diffusion signal can be expressed as a spherical convolution of the fiber orientations, or the FOD function, and a convolution kernel expressed as rotational harmonics\cite{Tournier_2004}. If both the FOD and the diffusion signal are expressed using SH function, as was the case for q-ball imaging, then the convolution can be expressed as a matrix multiplication, $\vec{c_s} = R\vec{c_{FOD}}$, where $\vec{c_s}$ and $\vec{c_{FOD}}$ are the SH coefficients for the diffusion signal and FOD function respectively and $R$ is the convolution kernel. Using the CSD model, the full equation for the diffusion signal $\vec{s}$ becomes:

\begin{equation}
\vec{s} = Z_gR\vec{c_{FOD}}
\end{equation}

The convolution kernel $R$ must be estimated from the response function, however if the response function is assumed to be axially symmetric, of equivalently the SH representation of the response function only contains nonzero coefficients for the $m = 0$ SH functions, then the convolution kernel is known to be diagonal and can be simply estimated by dividing the SH representation of the response function by the SH representation of a delta function \cite{Tournier_2007}.

The CSD method uses the constrained deconvolution to estimate the FOD from the diffusion signal. Because the $R$ matrix is easily invertible, the deconvolution of the diffusion signal can be expressed as a matrix multiplication $\vec{c_{ODF}} = R^{-1}\vec{c_s}$, where $\vec{c_s}$ is the least squares estimate of the SH representation of the diffusion signal. However, as is often the case with deconvolution problems, this inversion is ill-posed because inverting the $R$ matrix generally creates large terms which, when applied to noisy diffusion signals, magnify the signal noise. The CSD method proposes to overcome this issue by applying a positivity constraint during the deconvolution. This constraint requires that the FOD be positive on as much of the sphere as is possible. This constraint is biologically motivated because negative fiber densities do not have a meaningful interpretation. The original presentation of the positivity constraint by Tournier et al. proposes an iterative solution of the deconvolution, where the FOD is sampled on a discrete sphere. At each iteration, a regularization term is added to the deconvolution matrix for each negative FOD sample in the previous iteration. The iteration stops when updating the constraints does not reduce the number of negative points on the FOD. This solution does not guarinty a fully positive FOD, but it does allow a robust and reproducible solution to the deconvolution problem in the presence of noise. Later work has extended the CSD model and proposed a new solution method to the CSD problem that does not require a iterative regularization process.

%