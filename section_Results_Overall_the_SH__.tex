\section{Results}

Overall, the SH noise estimates and noise maps were the most accurate
and most precise. Figure \ref{fig:axEstimate} and \ref{fig:corEstimate} show a qualitative comparison of the
noise maps produced by the 4 single-acquisition methods along side the
repetition ground truth for human data at b-value of 3000. It is clear
from visual comparison of the four noise maps that the SH noise map (C)
most closely matches the ground truth (A). The DTI noise map (B)
captures some of the features present in the repetition ground truth,
however it overestimates the noise amplitude in several regions. The B0
noise map (D) is too grainy to capture any of the fine features in the
ground truth noise map. The B0 noise estimation method may still be
useful as a global estimate of the average noise amplitude in an image,
but it is not useful as a noise map. The noise maps produced using the
non-local means filtering (E) also had a tendency to overestimate the
noise specific tissue regions. In the next section, we present a
quantitative analysis of accuracy and precision of the 4
single-acquisition noise estimation methods.