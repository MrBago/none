\section{Results}

    In our initial analysis we checked to see whether age, gender, total gray and white matter volume, cortical and sub-cortical gray matter volume, average fractional anisotropy, and average mean diffusivity were good independent predictors of BMI. Table \ref{tab:basic} shows the results of this analysis. However, of the features we checked, only age, fractional anisotropy and mean diffusivity showed a statistically significant association with BMI. We fit a linear model using these three significant features, and found that FA and MD accounted for about 6\% of the observed variance in BMI, however age no longer showed a statistically significant association with BMI when modeled with FA and MD. These results are summarized in Talbe \ref{tab:linearmodel}. We also check to see the cortical thicknesses of the freesurfer defined cortical regions were good predictors of BMI. Table \ref{tab:fsThck} shows the association between regional thickness and age adjusted BMI. We found that a significant number of regions in the frontal lobe, parietal lobe and anterior cingulate cortex showed a positive association with BMI. Also, three temporal lobe regions regions showed a strong negative association with BMI.
    
    To further understand the associations between specific GM and WM regions and BMI we used a PLS regression model. PLS regression allows us to build a model with a large number of predictors, which may in themselves be largely correlated, without over-fitting the model. We fit PLS regression models with age, tract based diffusion metrics, and cortical thickness as predictors, five models in total. Table \ref{tab:PLSmodels} shows how well each of these models predicted BMI in our data set. The number of latent variables for the PLS model were determined and the model was fit using the training data set (all subjects in the S500 and previous releases). The R\textsuperscript{2} values for each of the models was determined using the testing data set (new subjects in the S900 release). The models with tract based diffusion metrics included in the predictors did substantially better than the models that did not. The best model without diffusion metrics included age and GM thicknesses as predictors, and this model had an R\textsuperscript{2} of .045. The model with only tract based diffusion metrics had an R\textsuperscript{2} of .2. We did not find evidence that including cortical thicknesses or age in the model improved predictive performance above only using tract based diffusion metrics.
    
	Figures \ref{fa_effect_map} and \ref{md_effect_map} show a spacial representation of the PLS regression model for predicting BMI from tract based diffusion metrics. These spacial maps are an average of normalized effect maps across subjects. The effect maps show how much each tract or WM region contributes to the predicted BMI produced by the PLS model. Appendix 1 gives a formal definition of the effect map and more detail about how the effect maps are computed. Figure \ref{fa_effect_map} shows the relative importance of FA values in the PLS model and Figure \ref{md_effect_map} similarly shows the relative importance of MD values. In the maps, red indicate a positive association with BMI and blue areas a negative association, however it is important to remember that these associations are relative to other regions. That is to say, the relative values within a subject are more important than the absolute FA and MD values in specific regions or tract. For example, the MD in the frontal aspect of the cinguate bundle has a large positive association with BMI while the MD in the temporal aspect of the same bundle has a large negative association. This pattern in the effect map tells us that the contrast between frontal and temporal MD in the cigulate bundle drives the BMI  prediction more than the average MD or FA value in that bundle.
	