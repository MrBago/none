\subsection{Human Data}

The results using real-world data largely mirror those we see in the
simulation. As shown in Table \ref{tab:humanEstimate} and Figure \ref{fig:axEstimate}, the noise estimate using
the SH model of order 6 is consistently more accurate and more precise
than that of the DTI model; the median estimate is closer to 1.0 and the
RMS error is lower. Unlike the simulation, accuracy of the estimate
obtained using the SH model does show some b-value dependence, but not
as much as the DTI model. The magnitude of this b-value dependence is
similar to that observed in the B0 noise estimate. The RMS error in the
voxelwise noise estimates obtained using the B0 method is substantially
higher than the RMS error associated with the DTI or SH models. While
the B0 method produces a fairly accurate estimate, the average estimated
noise amplitude is close to 1.0, the high RMS error means this method
does not produce useful noise maps (Figure \ref{fig:axEstimate} D). The noise maps produced
using non-local means filtering overestimated the noise in boundary
regions between tissues. The overestimate can be seen in the noise maps
and results in a higher RMS error compared to the ground truth noise
map. The SH noise estimates and noise maps were the most accurate and
most precise, but similar to the simulation results, the SH order
influences precision of SH noise estimates.

Figure \ref{fig:humanGraph} shows the accuracy and precision of the SH noise estimates as a
function of SH order. We see that the accuracy (Figure \ref{fig:humanGraph} blue line) of
the SH noise estimate does not change substantially once a sufficiently
high SH order is used. In order to measure the precision of the noise,
we use the repetition ground truth as a reference. We define the RMS
error to be the root mean square error between the noise estimate and
the noise amplitude measured from the repetition data set. Lastly, we
can estimate the stochastic component of the RMS error using
simulations. The RMS error of the noise estimates as a function of SH
order (Figure \ref{fig:humanGraph} green line) predicted the expected stochastic error
depicted by the red line. The RMS error of the noise estimates show a
minimum at SH order 4 for b-values 1000 and 2000, and a minimum at SH
order 6 for b-value of 3000. After the minimum, the RMS error increases
with SH order. The RMS error in the SH estimates (Figure \ref{fig:humanGraph} green line)
was higher than the stochastic RMS error we predicted using simulation
(Figure \ref{fig:humanGraph} red line). This means that not all the error in the noise
estimate can be explained by sampling variation, and other sources of
error are present in the estimate. However, the two lines show the same
trend and sampling variation does explain the increase in RMS error and
decrease in estimate precision as a function of SH order.

The precision difference between the noise estimates produced using low
and high order SH models can be seen by comparing the respective noise
maps. Figure \ref{fig:perSHOrder} shows axial and coronal slices from three noise maps
estimated using SH orders 4 (B), 14 (C) and 16 (D) compared to the
ground truth noise map generated from repeated measurements (A). The SH
order 6 (B) estimate is visibly smoother than higher order estimates and
more fine detail is visible in this lower order noise map. For example,
the outline of the Globus Pallidus (in the red box) is visible in the
ground truth noise map (A) and this structure has visibly lower noise
amplitude than the surrounding tissues. This structure has similar
detail in the SH order 6 estimate (B), but in the SH order 14 (C) and 16
(D) estimates the structure becomes progressively harder to resolve.
This is not surprising because the precision of the noise estimates
defines the level of contrast in the noise maps. The SH order 6, 14 and
16 estimates have 152, 60, and 27 degrees of freedom respectively. SH
harmonic order 8 is commonly used to fit SH functions to HARDI data. In
order to produce noise maps with a matching number of degrees of freedom
to B, C and D in Figure \ref{fig:perSHOrder} with a SH order 8 model, one would need to
acquire 198, 106, and 73 linearly independent diffusion gradient
directions respectively.