\subsection{Simulation Results}

In simulation, the SH noise estimates show near-perfect behavior, while
the DTI noise estimates are positively biased. Table \ref{tabIMsimEstimate} shows the average
noise estimates and root mean square (RMS) errors obtained from the
simulated data using the two residual bootstrap methods. The average
noise estimate is a measure of the method accuracy. The RMS error is a
measure that captures both the bias (the difference between the average
estimate and the expected result) and the precision (sampling error) in
an estimate. From this table we can see that, at least in simulation,
the accuracy and precision of the SH residual method is not dependent on
b-value or on SNR. The median noise estimate is close to 1.0 (the
simulated noise amplitude) at all three b-values and at both SNRs.
Additionally, the RMS error in the estimate is consistently 0.11 (the
analytical value we expect) and does not depend on b-value or SNR. The
DTI model does almost as well as the SH model at a b-value of 1000; the
noise estimates are 1.02 -- 1.04 and the RMS error of the voxelwise
estimate is 0.11 - 0.12. At higher b-values, however, the DTI model is
insufficient to represent the data and overestimates the noise. At a
b-value of 3000, for example, the median DTI estimates of noise
amplitudes are 1.19 and 1.41 at SNR 18 and 25 respectively.

Even though the SH noise estimation performs well in simulation, it is
important to consider the impact of the SH order on the quality of the
noise estimate. The noise estimate is a sample statistic that is
computed using a finite sample from the noise distribution. Assuming the
noise distribution is a standard normal distribution, the expected
variance of a noise amplitude estimate computed with\(\text{\ d}\)
degrees of freedom is \(\sqrt{\frac{2}{d}}\). This is the stochastic
component of the RMS error, or the RMS error one would expect purely
from random sampling effects (Figure \ref{fig:simGraph} red line). Because the number of
degrees of freedom in the SH noise estimate go down as a function of SH
order, we expect the RMS error to rise with SH order. Looking closely at
Figure \ref{fig:simGraph}, one can see that the RMS error in the SH estimates (Figure \ref{fig:simGraph}
green lines) closely match the analytically predicted stochastic RMS
error (Figure \ref{fig:simGraph} red line) begging with SH order 4. Where the two lines
match, the only source of error in the SH estimate is sampling. At SH
order 2, the model does not fit the data well and produces an
overestimate for the noise amplitude and a high RMS error. This effect
is larger at higher b-value because the signal is composed of more high
order SH content. In order to minimize RMS error in the noise estimate a
SH order must be chosen to balance accurate model fitting with
preserving more degrees of freedom for the noise estimate.
