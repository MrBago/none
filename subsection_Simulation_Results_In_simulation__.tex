\subsection{Simulation Results}

In simulation, the SH noise estimates show near-perfect behavior, while
the DTI noise estimates are positively biased. Table \ref{tabIMsimEstimate} shows the average
noise estimates and root mean square (RMS) errors obtained from the
simulated data using the two residual bootstrap methods. The average
noise estimate is a measure of the method accuracy. The RMS error is a
measure that captures both the bias (the difference between the average
estimate and the expected result) and the precision (sampling error) in
an estimate. From this table we can see that, at least in simulation,
the accuracy and precision of the SH residual method is not dependent on
b-value or on SNR. The median noise estimate is close to 1.0 (the
simulated noise amplitude) at all three b-values and at both SNRs.
Additionally, the RMS error in the estimate is consistently 0.11 (the
analytical value we expect) and does not depend on b-value or SNR. The
DTI model does almost as well as the SH model at a b-value of 1000; the
noise estimates are 1.02 -- 1.04 and the RMS error of the voxelwise
estimate is 0.11 - 0.12. At higher b-values, however, the DTI model is
insufficient to represent the data and overestimates the noise. At a
b-value of 3000, for example, the median DTI estimates of noise
amplitudes are 1.19 and 1.41 at SNR 18 and 25 respectively.

