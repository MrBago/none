\subsection{Simulation}

Model bootstrap methods work well when the model assumptions are
satisfied, but tend to overestimate residuals when the model assumptions
are violated. In this application, the extent to which model assumptions
are violated depends on the b-value, the SH order, and the SNR of the
data. In order to understand the performance of the two bootstrap
methods as a function of these choices, we used a simulation of two
equal-sized fibers crossing at 90 degrees. We simulated each of the two
fibers using a tensor model with FA of .7 and mean diffusivity of .5 $cm^2/sec$. To simulate Rican noise, we added samples from a
standard normal distribution (\(\mu = 0\), \(\sigma = 1\)) in both the
real and complex channels to this simulated signal, and then computed
the amplitudes. To vary the SNR, we scaled the signal and held the
simulated noise amplitude constant at 1. At each SNR and b-value, we
simulated 10,000 independent acquisitions. The simulations were done
using the same b-values and 181-direction gradient table, as were used
for acquiring the human subject data. The simulations for this study
were performed using the Diffusion Imaging in Python (Dipy) software
library \cite{Garyfallidis_2014}.

\subsection{Data collection}

A human subject was scanned on a 3T Siemens Skyra MRI scanner after
obtaining informed consent per the requirements of our local
institutional review board. The human subject was scanned to acquire 6
total data sets: a repetition data set and a HARDI data set at each
b-value (b=1000, 2000, and 3000 \(\frac{s}{mm^{2}}\)). For each b-value,
the repetition data set consisted of 3 gradient directions acquired 80
times each, along with 7 images with no diffusion weighting. Each HARDI
data set consisted of 181 non-collinear gradient directions and 7 images
with no diffusion weighting. All images were acquired at 2.2mm isotropic
resolution, TR = 3s, using a multiband factor of 2.

We used a motion and eddy current correction scheme similar to the
method employed by FSL's ``eddy\_correct''. This correction scheme
registers all the diffusion-weighted images to a reference to compensate
for scaling and shearing introduced by eddy and subject motion during
the scan. In order to compare voxelwise between the HARDI and repetition
data sets, we combined the b0 images from the two data sets to make one
reference image. This method produced one eddy correction reference for
each of the 3 b-values acquired. To produce this reference, we first
averaged all b0 images from the two data sets. We then registered those
same b0 images to the average image, and computed the average transform
to ensure that all images are moved as little as possible. We then used
this image as the reference for our motion and eddy current correction.
The end result of this process was a repetition and a HARDI data set
that were eddy-corrected and co-registered to each other. The repetition
data set was made up of repeated signal measurements along three
gradient directions. By taking the variance of the repeated measurements
at each voxel and averaging the three results, we created the ``ground
truth noise maps''. Noise maps were also estimated from the HARDI data
sets using each of the candidate methods described. Even though we do
not refer to the ground truth noise maps as estimates for brevity and
clarity, they are in fact estimates as well and when comparing the
candidate noise maps to the ground truth noise maps, one must account
for the standard error in both noise maps. For our quantitative
comparisons of the different noise estimates to the ground truth noise
maps, we limited the analysis to a brain tissue mask. The brain tissue
mask was created by first fitting a tensor model to each of the data
sets and taking the voxels with fractional anisotropy (FA) greater than
.1 and axial diffusivity less than 2 \(\frac{cm^{2}}{s}\). We then
performed binary erosion to the result to produce the final brain tissue
mask.