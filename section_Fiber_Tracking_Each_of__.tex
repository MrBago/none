\section{Fiber Tracking}

Each of the diffusion models previously described in this work allow us to model the directional alignment of white matter tissues and extract primary tract directions associated with those tissues. Whether the primary tract directions can be extracted directly, as is the case when the primary eigen direction of the tensor model is used, or indirectly, as is the case when peak finding is applied to an ODF or FOD, these directions serve as means to follow white matter pathways that connect distant brain regions. An early tracking approach was proposed by Mori et al., and dubbed  fiber assignment by continuous tracking or FACT \cite{Mori_1999}. The basic idea was to follow the path of white matter tissues by tracing along the primary direction estimated using a diffusion model. The FACT algorithm called for the tracing direction to be adjusted at every voxel edge. Figure \ref{fig:tracking} shows an example of such a procedure. More recently, tracking algorithms have generally switched to using a small, fixed step size and reevaluating the tracking direction after each step. Despite small changes, the basic idea hasn't changed all that much.